\chapter{Prototype}
\label{cha:prototype}

\section{Architecture}
\subsection{Overview}
\subsection{Messaging}
\subsubsection{Publish/Subscribe}
\subsubsection{Filtering}
\subsection{Coupling}
\subsection{State}

\section{Technology}
We have created a python application using Gevent\footnote{\url{http://www.gevent.org/}},
zeromq\footnote{\url{http://zeromq.org/}} and zerorpc\footnote{\url{http://zerorpc.dotcloud.com/}} 
to be able to service multiple requests in a non-blocking way. 

\subsection{ØMQ}
The main library that powers our prototype is called ØMQ or ZeroMQ~\cite{ZeroMQ}.
ZeroMQ is an asynchronous messaging library written in C with 
bindings for many languages including python. This library helps us to easily scale and use 
different programming paradigms such as publish-subscribe, request-replay and push-pull.

\section{Compoenents}
\subsection{Distributed Storages}
Our aim is to distribute the information about available datasets and operations at each node. To achieve this
we let our application to launch a number of communicators and publish information about it's data.
Other nodes in our network have to subscribes on other nodes, ZeroMQ allows us to subscribe
to multiple publishers, therefore each node can subscribe to other nodes. Nodes frequently get
\textbf{news} from other nodes, for example availability of certain datasets on a node, then it
can use publish-subscribe to get extra information on that particular subject.
\subsubsection{Operation Store}
\subsubsection{Dataset Store}
\subsection{Message Handlers}
\subsection{Decorators}
\subsection{Application}

\section{Layers}
\subsection{Network}
\subsubsection{API}
Since this is going to be a network program we need to use a form of Remote Procedure Call (RPC) 
to communicate between nodes. Rather than implementing ourselves we used a library based on zeromq 
called \textit{zerorpc}. Using this library we now expose a set of APIs and let the nodes talk to 
each other based on this API. There are multiple solutions for exposing services which we do not discuss here.
\subsubsection{Publisher}
\subsubsection{Listener}
\subsection{Pre-processing}
\subsubsection{Incoming Messages}
\subsubsection{API Calls}
\subsection{Backend}
\subsubsection{Logic}
\subsubsection{Stores}
\subsubsection{Database}

\section{Initialization}
First of all each application instance establish its own zeromq publisher socket. Then it subscribes
itself to all other nodes which are listed in config file.
\subsection{Local Database}
\subsection{Stores}
\subsection{Network}

\section{Control Flows}
\subsection{API Call Flow}
\subsubsection{Possible Reactions}
\subsection{Incoming Message Flow}
\subsubsection{Possible Reactions}


\section{Test Results}
To be able to assess the performance of each given solution to the mentioned scenarios we made a demo application
called \textbf{Konsensus} which its code is available on Github. \cite{konsensus}

\subsection{Testing Problems}
Writing tests for a distributed application is not as easy as writing unit-tests for a normal application. 
Our demo application acts as a server and client at the same time. Moreover we want to launch multiple 
network peers running on one or more machines. Testing scenarios on this network is not possible with
normal mocking approaches, because we need to test the behavior of our solution in a network of collaborating
peers which are not external, rather the core services of the application.

To overcome testing issues we have to launch the desired number of peers separately and then run our tests 
over them. To make this operation faster we changed the application to make it possible to launch any number
of instances on one machine and we automated this process using a number of scripts. %TODO more details

\subsubsection{Mixing Signals in Greenlets}
We use python Greenlets instead of threads. This means that our demo application runs on only one thread. 
This causes a problem when launching multiple apps all together with one script and inside one thread, that
causes the signals for events spread among all greenlets and make trouble. To avoid this we have to run
each server in a separate processes. Running them inside threads won't help as well because the blinker python
library is thread-safe so it moves signals between threads as well as Greenlets.

\subsubsection{Scnenario 2 Errors}
While testing scenario 2 we observed a common error. We this scenario with three different peers as the following table:

\begin{tabular}{ l c r }
\em{Server ID} & \em{ Dataset ID} & \em{ Client} \\
S0 & --- & Yes \\
S1 & DS1 & No \\
S2 & DS2 & No \\
\end{tabular}\\

We also used \textbf{Random Dataset Storage} mechanism, simply to store resulting dataset of one operation on one of the network
peers. The problem is when two peers decide to store the result of one operation on each other a blocking condition happens. Our
approach was opening a temporary port and inform the other peer to fetch the data. Meanwhile this is exactly happening on the other
peer, therefore both block.

\subparagraph{Solution}
To solve the blocking peers problem we used the already running main application API instead of opening temporary PULL/PUSH zeromq 
sockets. This change is working fine and the peers exchanging datasets with no problem.
