\chapter*{Preface}
\label{cha:preface}

European scientific communities launch many experiments everyday, resulting in huge amounts
of data. Specifically in molecular dynamics and material science fields there are many different
simulation softwares which are being used to accomplish multi-scale modeling tasks. 

These tasks
often involve running multiple simulation programs over the existing datasets or the data which is
produced by other simulation software. It's common to run multiple programs on existing datasets
during one operation to produce the desired results. The order to run simulation software is normally 
defined within scripts written by users. 

Moreover users have to provide the required data manually and copy all required files to a working
directory to submit their job, and they might have to login to different machines to prepare files,
submit the script, monitor the status of the job and finally collect the output files. 
This type of work routine is a common form of workflow management in above mentioned communities.

While simpler and smaller experiments could be handled this way, larger and more complicated experiments
require different solutions. 

Such experiments are the source of many high performance computing (HPC) problems, specially workflow management and data transfer.

%This thesis is an indirect research and development effort in the field of high performance computing (HPC) and scientific simulations.

This thesis is a research and development effort to accomplish such operations in a distributed manner with a 
collective but decentralized approach toward workflow management and to minimize data transfer during such operations.

This work has not been an implementation task nor a purely theoratical work. 
It means that I was not supposed to create an application or develop a software from ground up (even though eventually I did), instead
I have been responsible to study about and define the problem of my client and assist them either with 
finding a suitable solution and helping them to integrate it into their development process or propose a new approach to address their 
needs. My activities include but not limited to analysing the problem, collecting requirements,
studying state of the art software frameworks and related products,
analysing them against the defined requirments, proposing a solution and developing a prototype.

During this thesis an open source prototype application has been developed which is available
online\footnote{https://github.com/mehdisadeghi/konsensus}. The source files
of the current document are also available\footnote{https://github.com/mehdisadeghi/cme-thesis}.
If there are any comments and improvements regarding this document, I
appreciate an email to \textbf{sadeghi@mehdix.org}.

%\begin{center}
%\begin{tabular}{l}
%\nolinkurl{sadeghi@mehdix.org} \\
%Hochschule Offenburg\\
%Mehdi Sadeghi
%\end{tabular}
%\end{center}