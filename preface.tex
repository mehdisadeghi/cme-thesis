\chapter*{Preface}
\label{cha:preface}

European scientific communities launch many experiments everyday, resulting in huge amounts
of data. Specifically in molecular dynamics and material science fields there are many different
simulation softwares which are being used to accomplish multi-scale modeling tasks. These tasks
often involve running multiple simulation programs over the existing datasets or the data which is
produced by other simulation software. These programs often are coupled
together to produce the final result. Depending on the amount of the data and the desired type of simulation
these tasks could take many days to finish. The order to run simulation software and 
providing the input data are normally defined in scripts written by users which is the simplest form of workflow management.

While small experiments could be handled with simple scripts and normal computers, larger scale experiments
require different solutions. These type of tasks demand huge computing power which are made available by
computer clusters and super computers. An important characteristic of larger experiments is
the amount of produced data. This data, which needs to be transferred many times back and forth
between computers, grow by an order of magnitude, resulting in terabytes of data.

Large scientific experiments are the source of many high performance computing (HPC) problems, specially data transfer
and workflow management. Moreover HPC resources are expensive and should be used efficiently, therefore making data transfer
more efficient is important.

This thesis is an effort to know the main data transfer scenarios in the context of molecular dynamic use cases and try to address them in
a distributed manner with a collective but decentralized approach toward workflow management.

%The source files of the document are available online at this address:

%
%\begin{quote}
%\url{https://github.com/mehdisadeghi/cme-thesis}
%\end{quote}
%

If there are any comments and improvements regarding this document, the author
appreciates an email to the following address:

\begin{center}
\begin{tabular}{l}
\nolinkurl{sadeghi@mehdix.org} \\
Hochschule Offenburg\\
Mehdi Sadeghi
\end{tabular}
\end{center}