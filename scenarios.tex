\chapter{Scenarios}
\label{cha:scenarios}

There are a number of possible use cases in our problem domain. To demonstrate these cases we assume
we have a number of nodes and datasets respectively, but they are not necessarily on the same nodes.
In the following paragraphs we explain possible combinations of operations, nodes and datasets.

\section{Operation}
In every scenario we want to run an operation which could be linear or non-linear.

\subsection{Linear Operation}
Being linear means that the operation
could be broken into its components and then run in parallel or series. Here is algebic notation
of a linear operation which acts on two datasets:

\[ Operation(A + B) = Operation(A) + Operation(B) \]

Being linear or non-linear only matters when we have to operate on more that one dataset.

\subsection{Non-linear Operation}
In contrast to linear there is non-linear operation. This means that this kind of operation has dependant parts and
those parts could not run in parallel:

\[ Operation(A + B) \neq Operation(A) + Operation(B) \]

\section{Datasets}
For each operation we need one or more datasets which my be available on the same node that wants to run the operation
initially or could reside on other nodes. 

\subsection{Input}
Input files are normally not mission critical and could be reproduced.

\subsection{Output}
Operations create output datasets which normally are small in size, threfore we ignore the transfer cost of operation
results in our work.

\subsection{Data Locationing}
We consider three different approaches toward preparing required data for operations.
\subsubsection{Conventional Approach}
in this approach we put the required data on a network file system and all
application instances will access it there. We will utilize an NFS mounted file system.
\subsubsection{Centralized Approach}
in this approach we will have a central instance which will orchestrate operation
delegation and operation output forwarding to other nodes.
\subsubsection{Decentralized Approach}
in this approach we will eliminate the orchestrator node and the network of
application instances should collaborate in a decentralized fashion to keep track of data and control flow for each
task.

For every approach we will run performance tests and we will compare the results.

\subsubsection{Method}
We will discuss scenarios in chapter\ref{cha:scenarios}. For each scenario we will analyze the possible combinations 
of data and operations and we will discuss how to 
deliver the input data and where to store output data. We will discuss workflow management in chapter 
\ref{cha:workflow} and data transfer in chapter\ref{cha:data}.


% TODO: I might need to introduce "Decision Tree"
\section{Decision Making}
The main decision that we need to make at every scenario is whether we should transfer the required data or we
need to delegate the operation to an instance on a node which already has the data. To make a decision we need to
answer a number of questions. First we need to know the location of the data:

\begin{enumerate}
\item Is the data available locally?
\item If not, is the data available on another node? -- Here only the physical location of data matters not the instance
controlling it.
\end{enumerate}

% TODO: Introduce "Decision Metrics"

%In case the mentioned data is available on another nodes we have to answer these questions:
%\begin{enumerate}
%\item What is the cost of data transfer? -- We have to invent an algorithm for this calculation
%\item If data is available on more than one remote node, which one has the minimum transfer cost? -- We might
%introduce multiple strategies and use some heuristics for this selection, simplest form would be random selection,
%another could be asking for an availability metric from the instance and mix it with local calculated availability 
%metric to get a final cost value.
% TODO: We can use "Heartbeat" concept as one of availability metrics
%\item In case it is expensive to transfer the data, can we delegate the operation to an instance on the other node?
%\end{enumerate}

\subsection{Scenario 1}
In this scenario we have a linear operation, e.g. \(Op^A\) on \(Node^A\) which
requires \( Dataset^1 \).

\subsubsection{Conditions} \( Dataset^1 \) is not available on \( Node^A \).
\subsubsection{Consequences} With these conditions we either should transfer \( Dataset^1 \)
to local node or in case of availability delegate \(Op^A\) to the node which already has \( Dataset^1 \).